\documentclass[]{report}
\usepackage[utf8]{inputenc}
\usepackage[T1]{fontenc}
\usepackage[francais]{babel}
\usepackage{graphicx}
\graphicspath{{Figure/}} 
\begin{document}


\title{Mini projet 2 : Analyse statistique d’une famille de protéines}
\author{Fabien Tang\\
   Valentin Colliard}
\date{2018}
\maketitle
\setcounter{chapter}{1}
\section{Introduction}

Lors de ce projet, nous avons cherché à analyser statistiquement une famille de protéines donnée par un alignement de séquence en 
détectant :\\
- les positions conservées,\\
- les séquences appartenant à la même famille,\\
- les corrélations entre les différentes colonnes de l’alignement et leur relation avec les distances entres acides aminées.

\section{Données}\noindent
- Dtrain.txt contient M=5643 séquences de protéines d'une même famille.\\
Chaque séquence a pour longueur L= 48 positions et chaque acide aminé appartient à A = \{A, C, D, E, F, G, H, I, K, L, M, N, P, Q, R, S, T, V, W, Y,-\}.\\
- testseq.txt contient une séquence de longueur N = 114.\\
- distances.txt contient les distances entre paires d'acides aminées sous forme position 1, position 2, aa

\section{Modélisation par PSWM}
\subsection{Matrice de poids spécifiques des positions}
Dans un premier temps, nous avons chargé nos données dans python grâce aux fonctions de lecture de fichier
puis nous avons définis les fonctions n(i,a,liste) et w(i,a,liste) permettant respectivement de compter le nombre d'occurences 
et de calculer le poids d'une acide aminée a à une position i donné.\\
Les fonctions n\_global(liste) et w\_global(liste) utiliserons les fonctions précédentes pour la création des différentes matrices respectives
pour chaque position i = 0, ..., L-1 et chaque acide aminée a appartenant à A.
\subsection{Conservation}
Une fois les différentes matrices obtenus, nous avons codé la fonction s(i,liste) calculant l'entropie relative en fonction d'une position i.\\
Cette fonction est appelé par la fonction s\_global\_trie(liste) afin de déterminer les différentes positions
qui ont un poid très élevé pour une acide aminée.\\
ai(liste) se charge de determiner les 3 acides aminées les plus conservées.\\ 
\\
Les trois positions plus conservées et les acides aminées conservées:
\\
(31, 4.176827384837058), (46, 3.9862728301572674), (43, 3.875513964134912)\\
'W', 'P', 'G'\\

\includegraphics[scale=0.7]{Figure_1}
\subsection{Evaluer une nouvelle séquence}
Afin de décider si une nouvelle séquence fait partie de la même famille d'une protéine, 
\section{Conclusion}

\end{document}